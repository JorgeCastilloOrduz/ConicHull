
\documentclass[12pt, oneside]{article}%Tipo de documento y tamaño de letra%
\usepackage[spanish]{layout}
%\usepackage{hyperref}
\usepackage{color,graphicx}%Para poder incertar graficas%
\usepackage{graphics}
\usepackage{amsmath,amssymb}%Insertar unos simbolos matematicos especiales%
\usepackage{setspace}
%\usepackage{empheq}
%\usepackage{multicol}
\onehalfspacing
%\usepackage[mathscr]{euscript}%Tipo especial de letra%
\usepackage[utf8]{inputenc}%Para las tildes
\usepackage[spanish,activeacute]{babel} %Todo en Español
%\usepackage{multicol}
\pagestyle{empty}
%\usepackage{spalign}  %SISTEMAS DE ECUACIONES%
%\usepackage{array}
\usepackage{layout}
\usepackage{bm}
\usepackage{manfnt}
\usepackage{float}
\usepackage{enumitem}
%\usepackage{mma} 
\usepackage{xcolor}
\usepackage{mdframed}

\usepackage{hyperref}
\hypersetup{
    colorlinks=true,
    linkcolor=red}
\def\Car{{\textsf{Car}}}
\def\Ring{{\mathcal R}}

\def\N{{\mathbb N}}
\def\Z{{\mathbb Z}}
\def\R{{\mathbb R}}
\newtheorem{teorema}{Teorema}
\def\F{{\mathbb F}}
\def\multiset#1#2{\ensuremath{\left(\kern-.3em\left(\genfrac{}{}{0pt}{}{#1}{#2}\right)\kern-.3em\right)}}
\usepackage{picinpar}
\setlength{\oddsidemargin}{0pt}
\setlength{\topmargin}{0pt}
\setlength{\headheight}{0pt}
\setlength{\headsep}{0pt}
\setlength{\textheight}{24cm}
\setlength{\textwidth}{16.5cm}
\setlength{\marginparsep}{0pt}
\setlength{\marginparwidth}{0pt}
\setlength{\footskip}{1cm}

% \newenvironment{theorem}
%   {\begin{mdframed}[backgroundcolor=lightgray]
%   \begin{mdtheorem}}
%   {\end{mdtheorem}
% \end{mdframed}}


\begin{document}
\setlength{\parindent}{0cm}%EL ANCHO DE LA SANGRIA DE AQUÍ EN ADELANTE%
\hoffset-0.46cm
\voffset-1.46cm

\begin{window}[0,l,{\includegraphics[scale=0.4]{Logo UN.jpg}},]
\Large  \hspace{0.6cm}\textsf{National University of Colombia} \\
\textcolor{white}{\tiny.}  \Large \hspace{0.6cm} \textsf{Department of Mathematics} \\
\textcolor{white}{\tiny.}   \large\hspace{5.5cm}\textsf{Introduction to Optimization}\\
\textcolor{white}{\tiny.}   \large \hspace{6.05cm}\textsf{Linear Matrix Inequality} 
\end{window}


\vspace{0.5cm}
\normalfont
\textsf{Jorge Luis Castillo Orduz} 
\normalsize
\dotfill
\vspace{1cm}

Show that the solution set of a linear matrix inequality is convex.\\

\Large\textsf{\textcolor{blue}{Solution}}
\normalsize\\

Given a linear matrix polynomial $A:\R^n \rightarrow SM(\R^{mxm})$ defined by:
$$A(x)=A_0+x_1A_1+ \dots + x_n A-n $$

where $SM(\R^{mxm})$ is the set of $mxm$ symetric matrices of real numbers, and matrices $A_0,A_1,\cdots ,A_n$ are symetric as well. We can form the following linear matrix inequality $A(x)	\preceq B_m$
whose solution set is $S:=\{x\in \R^n\hspace{0.1cm} | \hspace{0.1cm}A(x)\preceq B_m \}$. Then, this set $S$ is the one to demonstrate is convex. Let $x,y\in S$ and $\lambda\in [0,1]$. Hence:
\begin{align*}
    A(\lambda x +(1-\lambda)y) &\preceq B_m\\
    A_0+(\lambda x_1 +(1-\lambda)y_1)A_1+ \dots + (\lambda x_n +(1-\lambda)y_n) A_n&\preceq B_m \\
    A_0+\lambda x_1A_1 +(1-\lambda)yA_1+ \dots + \lambda x_n A_n +(1-\lambda)yA_n&\preceq B_m \\
    A_0+\lambda(x_1A_1+\dots + x_n A_n) +(1-\lambda)(y_1A_1+ \dots +y_nA_n)&\preceq B_m \\
    \lambda A(x) +(1-\lambda)A(y) &\preceq B_m
\end{align*}
Since $x,y\in S$ then $A(x)\preceq B_m$ and $A(y)\preceq B_m$. Since a convex combination of two positive semidefinite matrices is also positive semidefinite, we conclude that set $S$ is convex.

\end{document}

